\section{Schlussfolgerung}
%Im Schlusswort wird der Ausgang des Projektes noch einmal revidiert und zusammengefasst. Dabei
%werden die H�he- und Tiefpunkte des Projektes noch einmal erw�hnt.
Easy-PID konnte durchaus erfolgreich abgeschlossen werden. Das Tool erf�llt dabei s�mtliche Anforderungen, welche an ein solches Tool gestellt werden. Des weiteren wurden einige weitere Funktionen, namentlich die Mini-Version, die Simulation der geschlossenen Regelstrecke und der Export der Berechnung als PDF, als Zusatzfunktionen im Programm implementiert. Eine Nachjustierm�glichkeit f�r die Reglerparameter wurde dabei bewusst nicht umgesetzt, jedoch kann mittels eines Schiebereglers die Phasengangmethode optimiert werden.

Der Benutzer kann die Parameter der Regelstrecke in der grafischen Benutzeroberfl�che eingeben, woraufhin Easy-PID die Reglerparameter mittels verschiedener Methoden berechnet und die entsprechenden Sprungantworten grafisch darstellt. Die einzelnen Methoden k�nnen dabei auch ausgeblendet werden. F�r die Phasengangmethode existiert ausserdem eine Trimm-Regler, mit welchem die Phasengangmethode angepasst werden kann. Mit diesen Funktionen erm�glicht es Easy-PID dem Benutzer, den idealen Regler zu dimensionieren.

Der Projektverlauf ist dabei nicht ohne Probleme verlaufen: Das zu Beginn des Projektes aufgestellte Klassendiagramm konnte beispielsweise nicht �bernommen werden, da gewisse sp�tere Aufgaben noch nicht erkannt wurden. Deswegen musste im sp�teren Projektverlauf ein neues Klassendiagramm erstellt werden, welches s�mtliche Aufgaben und Funktionen beachtet und somit gut umgesetzt werden konnte. \\
Ein weiteres Problem war die Umsetzung von Matlab-Code in Java. Viele Funktionen, die in Matlab bereits vorimplementiert sind, mussten f�r das Tool neu programmiert werden. Auch haben die vielen kleinen Unterschiede zwischen den beiden Programmiersprachen immer wieder zu Problemen gef�hrt.

Am Ende konnte das Projekt jedoch termingerecht und korrekt funktionierend abgeliefert werden. Trotzdem gibt es viele Features, welche in zuk�nftigen Projekten noch realisiert werden k�nnten. Eine M�glichkeit w�re das Importieren und Exportieren von Projekten. Dank dem Model-View-Controller Entwurfsmuster kann Easy-PID auch problemlos f�r eine Mobile-App adaptiert werden.

% - Speichern und �ffnen von Projekten
% - Mobile-App
% - 