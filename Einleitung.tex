\section{Einleitung}
Das Berechnen von Reglern ist oftmals eine schwierige Angelegenheit, die sowohl Erfahrung als auch Zeit ben�tigt. Zur richtigen Dimensionierung helfen entweder Faustformeln oder auch grafische Methoden, die jedoch von Hand ausgef�hrt werden m�ssen. Folglich ist die Berechnung entweder ungenau oder sehr aufw�ndig. Da es fragw�rdig ist, ob solche manuelle Methoden in einer Zeit von Matlab und weiteren Berechnungstools noch zeitgem�ss sind, wurde im Projekt 2 eine Software erstellt, welche Regler automatisch anhand der "`Phasengangmethode zur Reglerdimensionierung"' von Jakob Zellweger dimensionieren kann.

Das Hauptziel des Projektes ist, ein funktionierendes Programm zu erstellen, welches die Reglerberechnung automatisch nach der Phasengangmethode von Zellweger sowie weiteren Faustformeln dimensioniert. Zur korrekten Funktion werden sowohl richtig implementierte Berechnungen ben�tigt als auch die richtige Darstellung der Ergebnisse, beispielsweise mittels sinnvoll skalierten Graphen. Zus�tzlich wurden einige optionale Ziele definiert. Die wichtigsten davon sind die Simulation der geschlossenen Regelstrecke, eine Miniversion des Programms sowie eine M�glichkeit zum Nachjustieren der Eingabeparameter.

Easy-PID ist ein Programm, mit welchem Regler anhand der Parameter der Schrittantwort automatisch dimensioniert werden k�nnen. Es besitzt ein intuitiv zu bedienendes User-Interface, das neben den Ein- und Ausgabeparametern auch das dynamische Verhalten des Regelkreises grafisch darstellt. Die Regler werden durch verschiedene Methoden berechnet. Durch diese verschiedenen Dimensionierungen und der M�glichkeit, Parameter in Echtzeit zu manipulieren, kann ein optimaler Regler gefunden werden.

Dieser Bericht beschreibt die Probleml�sung aufgeteilt in die drei Teilbereiche Elektrotechnik, Programmierung und Validierung. Der elektrotechnische Teil behandelt die hergeleiteten Formeln f�r die Reglerdimensionierung und die Programmierung dieser Formeln in Matlab, w�hrend der zweite Teil den Aufbau der Benutzeroberfl�che und die Umsetzung der Matlab-Programmierung in Java beinhaltet. Die Validierung beschreibt den Aufbau und den Vorgang der Tests sowie deren Ergebnisse.