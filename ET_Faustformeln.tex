%\documentclass{fhnwreport} %
%\usepackage[ngerman]{babel}
%\usepackage[T1]{fontenc}
%\usepackage[latin1]{inputenc}
%\usepackage{tikz}
%\usepackage{amsmath}
%\usetikzlibrary{arrows}
%\usepackage{lmodern}   %Type1-Schriftart f�r nicht-englische Texte 
%
%\usepackage{listings}
%\lstset{language=Matlab}
%
%\usepackage{color}
%
%% Farben f�r Matlab-Listings
%\definecolor{hellgelb}{rgb}{1,1,0.85}     % Hintergrundfarbe
%\definecolor{colKeys}{RGB}{0,0,255}       % blau
%\definecolor{colIdentifier}{RGB}{0,0,0}	  % schwarz
%\definecolor{colComments}{RGB}{34,139,34} % gruen
%\definecolor{colString}{RGB}{160,32,240}  % violett
%
%\lstset{%
    %language=Matlab,%
    %%backgroundcolor={\color{hellgelb}},%
		%backgroundcolor={},%
    %basicstyle={\footnotesize\ttfamily},%
    %breakautoindent=true,%
    %breakindent=10pt,%
    %breaklines=true,%
    %captionpos=t,%
    %columns=fixed,%
    %%commentstyle={\itshape\color{colComments}},%
		%commentstyle={\color{colComments}},
    %extendedchars=true,%
    %float=hbp,%
    %frame=single,%
    %framerule=1pt,%
    %identifierstyle={\color{colIdentifier}},%
    %keywordstyle={\color{colKeys}},%
    %numbers=left,%
    %numbersep=1em,%
    %numberstyle={\tiny\ttfamily},%
    %showspaces=false,%
    %showstringspaces=false,%
    %stringstyle={\color{colString}},%
    %tabsize=4,%
    %xleftmargin=1em,%
    %xrightmargin=1em%
%} 
%
%
%\begin{document}

\subsection{Faustformeln}\label{FistFormula}

Zum Dimensionieren von verschiedenen Reglertypen gibt es eine Vielzahl von fixen Einstellregeln. Diese sind einfach anzuwenden, da sie keine besonderen Vorkenntnisse ben�tigen, jedoch ist deren Genauigkeit oftmals ungen�gend. Die f�r dieses Projekt relevanten Faustformeln sind:

\begin{itemize}
	\item Chien/Hrones und Reswick
	\item Oppelt
	\item Rosenberg
\end{itemize}

Diese Faustformeln wurden gew�hlt, da sie ebenfalls die Werte $Ks$, $Tu$ und $Tg$ verwenden, welche mittels Wendetangenten aus der Schrittantwort herausgelesen werden k�nnen, die uns vom Auftraggeber zur Verf�gung gestellt wurde.

Zur Verifizierung der Faustformeln wurden verschiedene Quellen verglichen und die Mehrheit wurde als richtig angesehen. %Quellenindex hinzuf�gen [3] [4] in Pflichtenheft

\subsubsection{Chien/Hrones und Reswick}
\paragraph{PI}
\begin{itemize}
	\item Aperiodischer Verlauf
	\begin{itemize}
		\item Gutes St�rverhalten
			
			\[
				Kr=0.6\cdot\frac{Tg}{Ks\cdot Tu}
			\]
						
			\[
				Tn=4\cdot Tu
			\]
			
			\item Gutes F�hrungsverhalten
			
			\[
				Kr=0.45\cdot\frac{Tg}{Ks\cdot Tu}
			\]
						
			\[
			Tn=1.2\cdot Tg
			\]
		
		\end{itemize}
	\item	20\% �berschwingen
		\begin{itemize}
			\item Gutes St�rverhalten
			
			\[
				Kr=0.7\cdot\frac{Tg}{Ks\cdot Tu}
			\]
			
			\[
			Tn=2.3\cdot Tu
			\]
			\item Gutes F�hrungsverhalten
			
			\[
				Kr=0.6\cdot\frac{Tg}{Ks\cdot Tu}
			\]
			
			\[
			Tn=Tg
			\]
		\end{itemize}
\end{itemize}
\paragraph{PID}
\begin{itemize}
	\item Aperiodischer Verlauf
	\begin{itemize}
		\item Gutes St�rverhalten
		\[
		Kr=0.95\cdot\frac{Tg}{Ks\cdot Tu}
		\]
		
		\[
		Tn=2.4\cdot Tu
		\]
		
		\[
		Tv=0.42\cdot Tu
		\]
		
		\item Gutes F�hrungsverhalten
		
		\[
		Kr=0.6\cdot\frac{Tg}{Ks\cdot Tu}
		\]
		
		\[
		Tn=Tg
		\]
		
		\[
		Tv=0.5\cdot Tu
		\]
		
	\end{itemize}
	\item 20\% �berschwingen
	\begin{itemize}
		\item Gutes St�rverhalten
		
		\[
		Kr=1.2\cdot\frac{Tg}{Ks\cdot Tu}
		\]
		
		\[
		Tn=2\cdot Tu
		\]
		
		\[
		Tv=0.42\cdot Tu
		\]
		
		\item Gutes F�hrungsverhalten
		
		\[
		Kr=0.95\cdot\frac{Tg}{Ks\cdot Tu}
		\]
		
		\[
		Tn=1.35\cdot Tg
		\]
		
		\[
		Tv=0.47\cdot Tu
		\]
		
	\end{itemize}
\end{itemize}

\subsubsection{Oppelt}
\paragraph{PI}
\[
Kr=0.8\cdot\frac{Tg}{Ks\cdot Tu}
\]

\[
Tn=3\cdot Tu		
\]
\paragraph{PID}

\[
Kr=1.2\cdot\frac{Tg}{Ks\cdot Tu}		
\]

\[
Tn=2\cdot Tu		
\]

\[
Tv=0.42\cdot Tu		
\]

\subsubsection{Rosenberg}
\paragraph{PI}

\[
Kr=0.91\cdot\frac{Tg}{Ks\cdot Tu}		
\]

\[
Tn=3.3\cdot Tu		
\]

\paragraph{PID}

\[
Kr=1.2\cdot\frac{Tg}{Ks\cdot Tu}
\]

\[
Tn=2\cdot Tu		
\]

\[
Tv=0.44\cdot Tu
\]
%\end{document}