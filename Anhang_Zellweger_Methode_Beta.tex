%\documentclass{fhnwreport} %
%\usepackage[ngerman]{babel}
%\usepackage[T1]{fontenc}
%\usepackage[latin1]{inputenc}
%\usepackage{tikz}
%\usepackage{amsmath}
%\usetikzlibrary{arrows}
%\usepackage{lmodern}   %Type1-Schriftart f�r nicht-englische Texte 
%
%\begin{document}
\section{Anhang Elektrotechnik}

\subsection{Herleitung der Berechnung von $\beta$ f�r die Zellweger Methode}\label{ZellwegerBeta}
F�r den offenen Regelkreis $Go$ gilt:

\begin{center}
	$\phi_{RE}$ Phase Regler

	$\phi_S$ Phase Strecke

	$\phi_O$ Phase offener Regelkreis
	\[
	\phi_{RE}(w_{PID})+\phi_S(w_{PID})=\phi_O(w_{PID})
	\]
\end{center}
F�r die Steigung (Ableitung) gilt:
\begin{equation}
\frac{d\phi_{RE}(w_{PID})}{dw}+\frac{d\phi_S(w_{PID})}{dw}=\frac{d\phi_O(w_{PID})}{dw}
\label{eq:zellwegerPhaseOffenerRegelkreis}
\end{equation}
Der offene Regelkreis berechnet sich wie folgt:
\[
Go=\frac{1}{s\cdot T0\cdot (1+s\cdot T1)}
\]
Die Phase von $O$ berechnet sich wie folgt:
\[
Tk=\frac{1}{w_{PID}}
\]
\[
\phi_O(w_{PID})=-\frac{\pi}{2}-\arctan \left(w\cdot \frac{1}{w_{PID}} \right)=
-\frac{pi}{2}-\arctan \left(w\cdot Tk \right)
\]
$\phi_O$ wird nach $w$ abgeleitet und f�r $Tk$ wird 
$\frac{1}{w_{PID}}$ eingesetzt. Da in $w_{PID}$ abgeleitet 
wurde wird $w$ gleich $w_{PID}$ gesetzt:
\begin{equation}
\frac{d\phi_O(w_{PID})}{dw}=-\frac{Tk}{1+w^{2}\cdot {Tk}^{2}}=-\frac{Tk}{1+{w_{PID}}^{2}\cdot \left( \frac{1}{w_{PID}} 
\right)^{2}}=-\frac{1}{2}Tk=-\frac{1}{2\cdot w_{PID}}
\label{eq:zellwegerPhaseAbleitung}
\end{equation}
Aus der Ableitung der Phase des offenen Regelkreises \eqref{eq:zellwegerPhaseAbleitung} und der Phasengleichung \eqref{eq:zellwegerPhaseOffenerRegelkreis} des offenen Regelkreises folgt:
\[
\phi_{RE}(w_{PID})+\phi_S(w_{PID})=-\frac{1}{2\cdot w_{PID}}
\]
Mit $w_{PID}$ multipliziert
\[
w_{PID} \cdot \frac{d\phi_{RE}(w_{PID})}{dw}+w_{PID} \cdot \frac{d\phi_S(w_{PID})}{dw}=-0.5
\]
Wie man einfach erkennt folgt daraus:
\[
w_{PID} \cdot \frac{d\phi_{RE}(w_{PID})}{dw}=\frac{2\cdot \beta}{1+\beta^{2}}
\]
Eingesetzt in die Phasengleichung:
\[
\frac{2\cdot \beta}{1+\beta^{2}}+w_{PID} \cdot \frac{d\phi_S(w_{PID})}{dw}=-0.5
\]
F�r $\phi_S(w_{PID})$ wird eingesetzt:
\[
\frac{2\cdot \beta}{1+\beta^{2}}+w_{PID} \cdot (-\arctan \left( w_{PID}\cdot T1 
\right)-\arctan \left( w_{PID}\cdot T2 \right)-\arctan \left( w_{PID}\cdot Tz \right))=-0.5
\]

Als Summe geschrieben:
\[
\frac{2\cdot \beta}{1+\beta^{2}}-w_{PID} \cdot  \sum \limits_{m=1}^{n} \arctan \left( w_{PID}\cdot T_m\right) =-0.5
\]

$\phi_S(w_{PID})$ abgeleitet:

\[
\frac{2\cdot \beta}{1+\beta^{2}}-w_{PID} \cdot \left(\frac{T_{1}}{1+ (w_{PID} \cdot T_{2})^2}+\frac{T_{2}}{1+ (w_{PID} \cdot T_{2})^2}+\frac{T_{m}}{1+ (w_{PID} \cdot T_{m})^2}\right)=-0.5
\]

Als Summe geschrieben:

\[
\frac{2\cdot \beta}{1+\beta^{2}}-w_{PID} \cdot  \sum \limits_{m=1}^{n} \frac{T_m}{1+ (w_{PID} \cdot T_m)^2} =-0.5
\]

Nun wird so umgeformt, dass alle $\beta$ auf einer Seite stehen:

\[
\frac{2\cdot \beta}{1+\beta^{2}}=w_{PID} \cdot  \sum \limits_{m=1}^{n} \frac{T_m}{1+ (w_{PID} \cdot T_m)^2} -0.5
\]

Die rechte Seite wird mit $Z$ substituiert:

\[
Z=w_{PID} \cdot  \sum \limits_{m=1}^{n} \frac{T_m}{1+ (w_{PID} \cdot T_m)^2} -0.5
\]

\[
Z=\frac{2\cdot \beta}{1+\beta^{2}}
\]

Nach $\beta$ aufgel�st:
\[
\beta = \frac{1}{Z} - \sqrt{\frac{1}{Z^2}-1}
\]

%\end{document}
