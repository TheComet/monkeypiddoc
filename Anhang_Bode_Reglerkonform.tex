%\documentclass{fhnwreport} %
%\usepackage[ngerman]{babel}
%\usepackage[T1]{fontenc}
%\usepackage[latin1]{inputenc}
%\usepackage{tikz}
%\usepackage{amsmath}
%\usetikzlibrary{arrows}
%\usepackage{lmodern}   %Type1-Schriftart f�r nicht-englische Texte 
%
%\begin{document}


\subsection[Herleitung Bodekonforme Darstellung]{Herleitung der Bodekonformen und Reglerkonformen Darstellung}\label{Bodekonf}

Regler k�nnen in verschieden Darstellungen abgebildet werden. Unterschieden werden hier die reglerkonforme und die bodekonforme Darstellung, wobei oftmals standartm�ssig die reglerkonforme Darstellung gew�hlt wird. Folgend sind beide Darstellungen von PI- und PID-Regler aufgef�hrt. Beim PID-Regler wird zudem die Beziehung zwischen den beiden Formen hergeleitet.

\subsubsection{PI-Regler}
Der PI-Regler hat die Parameter $Kr$ und $Tn$.
\par \textbf{Reglerkonform:} %use \par for next line, use \\ or \newline for a new line
\begin{equation}
G_{R}(s) = K_{R}\left(1+\frac{1}{s\cdot Tn}\right)
\end{equation}


\textbf{Bodekonform:}
\begin{equation}
G_{R}(s) = K_{R}\left(\frac{1+s\cdot Tn}{s\cdot Tn}\right)
\end{equation}

\subsubsection{PID-Regler}
\textbf{Reglerkonform:}
\par Der PID-Regler hat in der reglerkonformen Darstellung die Parameter $Kr$, $Tn$, $Tv$ und $Tp$.
\begin{equation}
G_{R}(s) = K_{R}\left(1+\frac{1}{s\cdot Tn}+\frac{s\cdot Tv}{1+s\cdot Tp}\right)
\end{equation}
Umformung:

\begin{equation}
\begin{aligned} % use {aligned} and &= to aligne vertical all = 
	G_{R}(s) &= K_{R}\left(\frac{s \cdot Tn(1+s \cdot Tp) + (1+s \cdot Tp)(s \cdot Tn \cdot s \cdot Tv)}{s \cdot Tn(1+s \cdot Tp)}\right) \\ 
	&= K_{R}\left(\frac{s^2 \cdot Tn\cdot Tp + s^2 \cdot Tn \cdot Tv + s \cdot Tp + s \cdot Tn + 1}{s \cdot Tn(1+s \cdot Tp)}\right) \\
	&= K_{R}\left(\frac{s^2(Tn\cdot Tp + Tn \cdot Tv) + s(Tp + Tn) + 1 }{s \cdot Tn(1+s \cdot Tp)}\right) 
	\\
	&= K_{R}\left(\frac{s^2\cdot Tn(Tp + Tv) + s(Tp + Tn) + 1 }{s \cdot Tn(1+s \cdot Tp)}\right)
\end{aligned}
\end{equation}


\textbf{Bodekonform:}
\par Der PID-Regler hat in der bodekonformen Darstellung die Parameter $Krk$, $Tn$, $Tv$ und $Tp$.
\begin{equation}
G_{R}(s) = K_{RK}\left(\frac{(1+s\cdot Tnk)(1+s\cdot Tvk)}{s\cdot Tnk(1+s\cdot Tp)}\right)
\end{equation}
Umformung:

\begin{equation}
\begin{aligned} 
G_{R}(s) = K_{RK}\left(\frac{s^2(Tnk \cdot Tvk)+ s(Tnk + Tnk) + 1}{s\cdot Tnk(1+s\cdot Tp)}\right)
\end{aligned}
\end{equation}

\par \textbf{Koeffizienten-Vergleich:}
\\Damit ein Koeffizienten-Vergleich zwischen der reglerkonformen und der bodekonformen Darstellung des PID-Reglers durchgef�hrt werden kann, m�ssen die beiden Formen gleichgesetzt werden (links reglerkonform, rechts bodekonform):

\begin{align}
K_{R}\left(\frac{s^2\cdot Tn(Tp + Tv) + s(Tp + Tn) + 1 }{s \cdot Tn(1+s \cdot Tp)}\right) &= K_{RK}\left(\frac{s^2(Tnk \cdot Tvk)+ s(Tnk + Tnk) + 1}{s\cdot Tnk(1+s\cdot Tp)}\right) \nonumber \\  %\textcolor[rgb]{1,0,0}{s}
\intertext{$s$  wird auf beiden Seiten im Nenner gek�rzt:} \nonumber
K_{R}\left(\frac{s^2\cdot Tn(Tp + Tv) + s(Tp + Tn) + 1 }{Tn(1+s \cdot Tp)}\right)&= K_{RK}\left(\frac{s^2(Tnk \cdot Tvk)+ s(Tnk + Tnk) + 1}{Tnk(1+s\cdot Tp)}\right) \\
\intertext{L�sst man die Frequenz $w$ gegen Null gehen, strebt auch $s$ gegen Null und die Terme lassen sich wie folgt vereinfachen:} \nonumber 
K_{R}\left(\frac{ 1 }{Tn}\right)&= K_{RK}\left(\frac{1}{Tnk}\right)
\end{align}

An der oben stehenden Umformung erkennt man, dass man einen Koeffizientenvergleich machen kann. Folgend werden zuerst die Koeffizienten mit $s$, danach die $s^2$ und am Ende die Koeffizienten ohne $s$ verglichen und nach den Parameter der reglerkonformen Darstellung umgeformt.

2. Koeffizient
\begin{equation}
\begin{aligned}
s(Tn+Tp) &= s(Tnk+Tvk) \\
Tn+Tp &= Tnk+Tvk \\
Tn &= Tnk+Tvk-Tp\\
\label{eq:2koeffizient}
\end{aligned}
\end{equation}

Damit ist $Tn$ bestimmt.

1. Koeffizient:
\begin{equation}
\begin{aligned}
s^2(Tn\cdot Tp + Tn \cdot Tv)&=s^2(Tnk \cdot Tvk)\\
(Tn\cdot Tp + Tn \cdot Tv)&=(Tnk \cdot Tvk)\\
Tn(Tv + Tp)&=Tnk \cdot Tvk
\end{aligned}
\end{equation}

Obenstehende Gleichung wird nun nach $Tv$ umgeformt:

\begin{equation*} %star after equations means NO automatic numbering
\begin{aligned}
Tv + Tp &= \frac{Tnk \cdot Tvk}{Tn}\\
Tv &= \frac{Tnk \cdot Tvk}{Tn} - Tp
\end{aligned}
\end{equation*}

Die Variable $Tn$ wird in der folgenden Gleichung durch die Beziehung von Formel \ref{eq:2koeffizient} ersetzt:

\begin{equation}
\begin{aligned}
Tv &= \frac{Tnk \cdot Tvk}{Tnk+Tvk-Tp} - Tp
\end{aligned}
\end{equation}

Damit ist $Tv$ bestimmt. $Kr$ wird wie folgt bestimmt:

3. Koeffizient:
\begin{align}
\frac{Kr}{s \cdot Tn (1+s\cdot Tp)} &= \frac{Krk}{s\cdot Tnk(1+s\cdot Tp)}\nonumber \\  %\nonumber means no number on this line
\intertext{$s$ und $(1+s\cdot Tp)$ werden gek�rzt und $Tn$ ersetzt:}
\frac{Kr}{Tnk+Tvk-Tp} &= \frac{Krk}{Tnk}\nonumber \\ 
Kr &= \frac{Krk(Tnk+Tvk-Tp)}{Tnk}\nonumber \\ 
\intertext{Das im Verh�ltnis zu $Tvk$ ca. zehn mal kleinere $Tp$ kann vernachl�ssigt werden:}
Kr &= \frac{Krk(Tnk+Tvk)}{Tnk} \nonumber \\ 
Kr &= \frac{Krk(1+Tvk)}{Tnk}
\end{align}

%\begin{equation}
%Tn = \frac{Tnk \cdot Tvk}{Tvk + Tp}
%\end{equation}



%Jetzt wird $Tn$ in $Tv$ eingesetzt:
%\begin{equation}
%\begin{aligned}
%Tv &= \frac{Tnk \cdot Tvk (Tv + Tp}{Tnk \cdot Tvk} - Tp
%\end{aligned}
%\end{equation}



%\end{document}

