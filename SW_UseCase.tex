\subsection{Ablauf einer Reglerberechnung (Use-Case)}
Im folgenden Abschnitt wird die Dimensionierung eines Reglers mit Easy-PID an einem Beispiel erkl�rt. Als Beispiel dient eine Regelstrecke mit den Kenngr�ssen $Tu=2$, $Tg=6$ und $Ks=1$, die parasit�re Zeitkonstante $Tp$ wird bei 10\% belassen, was der Standardeinstellung entspricht. Gew�nscht ist ein PID-Regler mit einem maximalen �berschwingen von 4.6\%.

\textbf{1.} \\
Die Werte f�r $Tu$, $Tg$ und $Ks$ werden in den entsprechenden \textit{JFormattedDoubleTextField} eingegeben, f�r $Tp$ muss nichts eingegeben werden, da 10\% bereits die Standardeinstellung ist. Mit den beiden Dropdown-Men�s, welche mittels \textit{JComboBox} implementiert wurden, wird ausserdem der Reglertyp PID und das �berschwingen auf 4.6\% festgelegt.

\textbf{2.} \\
Durch Dr�cken der Schaltfl�che \textit{Simulieren} wird die Methode \textit{actionPerformed} aufgerufen, welche die Eingabeparameter in die entsprechenden Attribute speichert. Die Eingabeparameter werden im \textit{View} saniert, damit ung�ltige Einstellung m�glichst fr�h erkannt werden k�nnen. Ist alles in Ordnung, wird die Methode \textit{Controller::btSimulateAction} aufgerufen. Bei nicht akzeptierten Eingabeparametern erscheint unter den Eingabefeldern ein \textit{JLabel} mit der entsprechenden Fehlermeldung. Bei akzeptierten Eingabeparametern wird die Simulation gestartet.

\textbf{3.} \\
Der \textit{Controller} konfiguriert das \textit{Model}-Objekt mit den Eingabewerten mittels der Methoden \textit{Model::set\-Regulator\-Type()} - welche den Typ zu \textit{I}, \textit{PI} oder \textit{PID} setzt, \textit{Model::setPlant()} - welche ein neues \textit{Plant}-Objekt erstellt und als Attribut der Klasse \textit{Model} speichert, \textit{Model::set\-Parasitic\-Time\-ConstantFactor()} und \textit{Model::setOvershoot()}. Danach ruft sie die Methode \textit{Model::simulateAll()} auf, um die Simulation zu beginnen.

\textbf{4.} \\
Das \textit{Model} baut sich mit Hilfe seines \textit{Plant}-Objekts, welches die vom Benutzer eingegebene Strecke beschreibt, eine Liste von \textit{CalculationCycle}-Objekten. Diese Objekte sind dazu f�hig, eine gesamte Berechnung von der Reglerberechnung bis zur Schrittantwort durchzuf�hren. Welche Regler ausgerechnet werden ist abh�ngig von der Auswahl des Reglertyps, also \textit{I}, \textit{PI} oder \textit{PID}.

Es wird mittels \textit{Model::notifySimulationBegin()} allen Listener mitgeteilt, dass eine Simulation beginnt. Dies bewirkt unter anderem dass sich die verschiedene Panels auf die Resultate vorbereiten k�nnen, dies umfasst beispielsweise die Tabelle oder den Plot zu l�schen.

Die \textit{CalculationCycle}-Klasse erbt von \textit{Runnable}. Es wird ein \textit{ThreadPool} erstellt und alle \textit{CalculationCycle}-Objekte werden parallel ausgef�hrt. Das Programm wartet, bis alle Berechnungen vollendet sind.

\textbf{5.} \\
Zu diesem Zeitpunkt teilt sich der Programmfluss in einer Unmenge kleiner Teilchen auf. Wir verfolgen nur einen Pfad, n�mlich die Berechnung eines PID-Reglers mittels Zellweger Methode.

Beim Instanziieren der Klasse \textit{CalculationCycle} wird ein \textit{AbstractControllerCalculator} �bergeben. In diesem Fall ist das ein Objekt, dass als konkrete Klasse den \textit{ZellwegerPID} hat. Dieses Objekt wurde schon vom \textit{Model} konfiguriert und es muss nur \textit{AbstractControllerCalculator::run()} aufgerufen werden, um einen passenden Regler f�r die vom Benutzer definierte Strecke zu berechnen. 

\textbf{6.} \\
Die \textit{run()}-Methode f�hrt dazu, dass die Methode \textit{ZellwegerPID::calculate()} ausgef�hrt wird. Der implementierte Algorithmus von Herrn Zellweger wird in dieser Methode gebraucht, um ein neues \textit{ControllerPID}-Objekt zu erstellen. Dieses Objekt beschreibt den Regler Anhand der Reglerparametern $T_n$, $T_v$, $K_r$ und $T_p$ sowie einer �bertragungsfunktion mit der Klasse \textit{TransferFunction}. Weiter berechnet das \textit{ZellwegerPID}-Objekt einen Minimal- und Maximalwert f�r $K_r$ aus, was sp�ter f�r das Approximieren des �berschwingens gebraucht wird.

\textbf{7.} \\
Der Regler ist berechnet und kann in der \textit{CalculationCycle} mit \textit{AbstractControllerCalculator::getRegulator()} als \textit{AbstractController} geholt werden. Der Regler wird zusammen mit dem \textit{Plant}-Objekt zu einem \textit{ClosedLoop}-Objekt zusammengef�gt. Das \textit{ClosedLoop}-Objekt berechnet aus der Strecke und dem Regler seine eigene \textit{TransferFunction} (�bertragungsfunktion).

\textbf{8.} \\
In der \textit{CalculationCycle} wird nun die Schrittantwort mit der Methode \textit{ClosedLoop::calculate\-Step\-Response()} ausgerechnet. Diese Methode Berechnet mittels Partialbruchzerlegung die Schrittantwort und misst auch gleich das maximale �berschwingen. Wenn der \textit{AbstractController} ein Minimal- und Maximalwert f�r $K_r$ ausgerechnet hat, was bei allen Zellweger der Fall ist, dann wird die Schrittantwort mehrmals ausgerechnet, bis das �berschwingen, welches mit der Methode \textit{ClosedLoop::getOvershoot()} geholt werden kann, mit dem vom Benutzer definierten Wert �bereinstimmt.

\textbf{9.} \\
Ist die \textit{CalculationCycle} abgeschlossen, wird mittels der Methode \textit{CalculationCycle::notify\-Step\-Response\-Calculation\-Complete()} allen Listener mitgeteilt, dass die Berechnung einer Schrittantwort fertig ist. Dabei werden die Reglerparameter $K_r$, $T_n$, $T_v$, $T_p$ und das �berschwingen in einer Tabelle im \textit{OutputPanel} eingetragen und die Schrittantwort wird in einem 2D-Plot auf der rechten Seite des GUIs in der Klasse \textit{GraphPanel} gezeichnet.

\textbf{10.} \\
Wenn der Threadpool keine Jobs mehr hat, dann wird allen Listener mittels der Methode \textit{Model::notifySimulationComplete()} mitgeteilt, dass alle Berechnungen beendet sind.

\textbf{11.} \\
Im \textit{GraphDisplayPanel} kann mittels der Checkboxen ausgew�hlt werden, welche Schrittantworten visualisiert werden k�nnen, um so einen m�glichst passenden provisorischen Regler zu finden. In unserem Beispiel ist dies der Graph des Reglers, welcher mit der Phasengang-Methode dimensioniert wurde.
%Bitte mit fachlichem Teil erg�nzen.

\textbf{12.} \\
Der provisorisch ausgesuchte Regler soll nun weiter verbessert werden. Dazu wird der Regler nach der Phasengang-Methode in der \textit{JTable}-Tabelle im \textit{OutputPanel} ausgew�hlt und anschliessend auf den \textit{JButton} �bernehmen gedr�ckt, wodurch der \textit{ActionListener} mit der \textit{Controller}-Methode \textit{btAdoptAction} aktiviert wird.
%Bitte mit fachlichem Teil erg�nzen.

\textbf{13.} \\
Mittels der \textit{JSlider} k�nnen nun die Reglerparameter angepasst werden, wobei sich der Graph im \textit{GraphPanel} in Echtzeit den Einstellungen anpasst.