\subsection{Controller}
%Problem/Fragestellung:
%- Welche Methoden muss der Controller enthalten?
%L�sung: Model und View bestimmen, welche Methoden der Controller enthalten muss.
Der $Controller$ ist sehr schlank gehalten und ist ein Teil des Model-View-Controller-Pattern. Der Controller hat dabei mehrere Aufgaben:
\begin{itemize}
	\item Beim Dr�cken des Simulationsschaltfl�che im \textit{InputPanel} werden die entsprechenden Methoden des Models aufgerufen, um die neuen Regler mit den neuen Eingabeparametern zu berechnen, die alten Simulationen zu l�schen und neue Simulationen zu starten. 
	\item Wenn der Trimm-Slider im \textit{OutputPanel} ver�ndert wird, so wird die entsprechende Methode im Model aufgerufen. 
	\item Wenn der Anwender eine Checkbox im \textit{GraphDisplayPanel} anklickt, so wird die entsprechende Methode im Model aufgerufen, um eine Kurve ein- oder auszublenden.  
\end{itemize}


%TODO Murray 
%Sehr viele 
%- Korrigieren und ev. erg�nzen.
%- Falls m�glich Text mit weiterem Inhalt erg�nzen. Ev. die Methoden genauer beschreiben und auf die entsprechenden Methoden des Models referenzieren.

%
%\subsection{Ablauf einer Reglerberechnung}
%Im folgenden Abschnitt wird die Dimensionierung eines Reglers mit Easy-PID an einem Beispiel erkl�rt. Als Beispiel dient eine Regelstrecke mit den Kenngr�ssen $Tu=??$, $Tg=??$ und $Ks=??$, die parasit�re Zeitkonstante $Tp$ wird bei 10\% belassen, was der Standardeinstellung entspricht. Gew�nscht ist ein PID-Regler mit einem maximalen �berschwingen von 4.6\%.
%
%\textbf{1.} \\
%Die Werte f�r $Tu$, $Tg$ und $Ks$ werden in den entsprechenden $JFormattedDoubleTextField$ eingegeben, f�r $Tp$ muss nichts eingegeben werden, da 10\% bereits die Standardeinstellung ist. Mit den beiden Dropdown-Men�s, welche mittels $JComboBox$ implementiert wurden, wird ausserdem der Reglertyp PID und das �berschwingen auf 4.6\% festgelegt.
%
%\textbf{2.} \\
%Durch Dr�cken der Schaltfl�che $Simulieren$ wird die Methode $actionPerformed$ aufgerufen, welche die Eingabeparameter in die entsprechenden Attribute speichert und die Methode $bt\-Simulate\-Action$ des $Controllers$ aufruft, welche wiederum die Methoden des Models aufruft. Bei nicht akzeptierten Eingabeparametern erscheint unter den Eingabefeldern ein $JLabel$ mit der entsprechenden Fehlermeldung. Bei akzeptierten Eingabeparametern wird die Simulation gestartet und es werden mit den Faustformeln und der Phasengangmethode die entsprechenden Regler berechnet. Die Schrittantworten der jeweiligen Regler erscheint im $GraphPanel$ auf der rechten Seite.
%
%\textbf{3.} \\
%Im $GraphDisplayPanel$ kann mittels der Checkboxen ausgew�hlt werden, welche Schrittantworten visualisiert werden k�nnen, um so einen m�glichst passenden provisorischen Regler zu finden. In unserem Beispiel ist dies der Graph des Reglers, welcher mit der Phasengang-Methode dimensioniert wurde.
%%Bitte mit fachlichem Teil erg�nzen.
%
%\textbf{4.} \\
%Der provisorisch ausgesuchte Regler soll nun weiter verbessert werden. Dazu wird der Regler nach der Phasengang-Methode in der $JTable$-Tabelle im $OutputPanel$ ausgew�hlt und anschliessend auf den $JButton$ �bernehmen gedr�ckt, wodurch der $ActionListener$ mit der $Controller$-Methode $btAdoptAction$ aktiviert wird.
%%Bitte mit fachlichem Teil erg�nzen.
%
%\textbf{5.} \\
%Mittels der $JSlider$ k�nnen nun die Reglerparameter angepasst werden, wobei sich der Graph im $GraphPanel$ in Echtzeit den Einstellungen anpasst.
%%Bitte mit fachlichem Teil erg�nzen.