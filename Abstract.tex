\documentclass[a4paper]{fhnwreport} %Legt grundlegende Formatierungen wie Schriftarten, Ort Seitenzahlen etc. fest.

%-----------------------------Wichtigste/Zentrale Packages------------------------ 
\usepackage[german]{babel}
%\usepackage[latin9]{inputenc}
\usepackage[babel, german=quotes]{csquotes}
\usepackage{hyperref}
\usepackage{verbatim}
\usepackage{amsmath}		%Mathe-Package
\usepackage{amsthm}			%Mathe-Package
\usepackage{graphicx} 	%Paket f�r die Darstellung von Abbildungen
\usepackage{fancyhdr}
\usepackage[latin1]{inputenc}
\usepackage[T1]{fontenc}
\usepackage{pdfpages}
\usepackage{tikz}
\usetikzlibrary{arrows}
\usepackage{lmodern}
\usepackage{listings}
\lstset{language=Matlab}



% Farben f�r Matlab-Listings
\definecolor{hellgelb}{rgb}{1,1,0.85}     % Hintergrundfarbe
\definecolor{colKeys}{RGB}{0,0,255}       % blau
\definecolor{colIdentifier}{RGB}{0,0,0}	  % schwarz
\definecolor{colComments}{RGB}{34,139,34} % gruen
\definecolor{colString}{RGB}{160,32,240}  % violett

\lstset{%
    language=Matlab,%
    %backgroundcolor={\color{hellgelb}},%
		backgroundcolor={},%
    basicstyle={\footnotesize\ttfamily},%
    breakautoindent=true,%
    breakindent=10pt,%
    breaklines=true,%
    captionpos=t,%
    columns=fixed,%
    %commentstyle={\itshape\color{colComments}},%
		commentstyle={\color{colComments}},
    extendedchars=true,%
    float=hbp,%
    frame=single,%
    framerule=1pt,%
    identifierstyle={\color{colIdentifier}},%
    keywordstyle={\color{colKeys}},%
    numbers=left,%
    numbersep=1em,%
    numberstyle={\tiny\ttfamily},%
    showspaces=false,%
    showstringspaces=false,%
    stringstyle={\color{colString}},%
    tabsize=4,%
    xleftmargin=1em,%
    xrightmargin=1em%
} 


%-----------------------------(Optionale) Pakete
\usepackage{subfigure}	%Paket f�r die Darstellung zweier Abbildungen �ber oder nebeneinander
\usepackage{booktabs}		%Paket f�r professionelle Tabellen
\usepackage{todonotes}	%Paket f�r die Nutzung von Randnotitzen. 
\usepackage{cite}				%Paket f�r die Zitation von Quellen
%\uspackage{listings}		% Nicht erw�hnt in Pr�si. N�tzliches Paket f�r die Dokumentierung von Software-Code

\bibliographystyle{IEEEtran}	%Legt den Bibliographiestyle fest (Harvard, IEEE etc.)
\graphicspath{{./graphics/}{./Anhang/}}	%Legt fest in welchen Ordnern Latex nach Bilder suchen soll (./ <- aktuelles Verzeichnis). Wichtig f�r die Strukturierung von Dokumenten
\geometry{twoside=false}	% F�r den einseitgen Druck. Falls doppelseitig gew�nscht: Befehl l�schen
\overfullrule=1em	%Erzeugt schwarzen Balken falls Text �ber den Rand l�uft (wenn Latex nicht autom. Silbentrennung macht). Vorallem f�r Debugging interessant.
\pagestyle{empty}
\begin{document}

\section*{Abstract}
%Was war die Aufgabe?
Um in der Praxis einen geeigneten Regler zu dimensionieren, ist oftmals viel Zeit und Erfahrung notwendig, was es schwierig macht, einen idealen Regler zu finden. Zwar gibt es zur Hilfe Faustformeln, diese liefern jedoch oftmals nur ungen�gende Ergebnisse, sodass man schliesslich auf grafische Methoden mit Papier und Bleistift zur�ckgreifen muss. Dies kann jedoch im 21. Jahrhundert nicht mehr als zeitgem�ss betrachtet werden. "`Easy-PID"' ist ein Softwaretool, welches im Rahmen des Projekt 2 der Fachhochschule Nordwestschweiz entwickelt wurde, das die Reglerdimensionierung mit der grafischen Phasengangmethode von Zellweger sowie mehreren Faustformeln automatisch durchf�hrt und so hilft, einen idealen Regler zu finden. Mit Hilfe der Kenngr�ssen der Regelstrecke sowie weiteren Einstellungen wie beispielsweise dem maximalen �berschwingen und der Wahl des Reglertyps kann "`Easy-PID"' alle ben�tigten Reglerparameter bestimmen.

%Wie wurde sie gel�st?
Zur richtigen Reglerdimensionierung mussten zuerst die Formeln zur Reglerdimensionierung und der Sprungantwort hergeleitet werden. Die damit erstellten Berechnungsalgorithmen wurden in Matlab implementiert und die Ergebnisse mit gegebenen Werten verglichen, um die Korrektheit zu �berpr�fen. Um "`Easy-PID"' flexibel in Java programmieren zu k�nnen, wurde auf das Model-View-Controller Entwurfsmuster zur�ckgegriffen. Die von "`Easy-PID"' gelieferten Ergebnisse wurden ausserdem mit Matlab verifiziert. \\
Das Java-Programm wurde anschliessend weiter optimiert, um ein ideales Verh�ltnis zwischen Geschwindigkeit und Genauigkeit zu finden. So wurden beispielsweise zur Berechnung der Schrittantworten zwei Vorgehensweisen implementiert: Residuen und IFFT. Dabei hat sich gezeigt, dass mit Residuen sowohl genauere als auch schnellere Resultate erzielt werden k�nnen.

%Welche Resultate wurden erzielt?
"`Easy-PID"' kann durchaus als erfolgreiches Projekt bezeichnet werden. Das Programm bietet nebst dem Berechnen der Reglerparameter aufgrund der Eingabeparameter und der Darstellung der Sprungantwort diverse Zusatzfunktionen wie eine Mini-Version, den Export als PDF oder die nachtr�gliche Anpassung der Phasengang-Methode. Der User kann zus�tzlich mit einem Hilfe-Men� hilfreiche Links aufrufen, die auftretende Fragen zum Thema Regelungstechnik beantworten k�nnen. \\
S�mtliche Daten werden dabei in einem einfach zu bedienenden User-Interface dargestellt. So kann der Benutzer die Eingabeparameter in Textfeldern eingeben und mittels einem Dropdown-Men� den Reglertyp w�hlen. Die mit diesen Daten berechneten Regler werden tabellarisch dargestellt. Die Sprungantwort der geschlossenen Regelstrecke wird ausserdem grafisch dargestellt, wobei auch einzelne Graphen ausgew�hlt werden k�nnen.

\end{document}